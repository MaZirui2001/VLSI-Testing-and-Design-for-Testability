%!TeX program = xelatex
\ExplSyntaxOn
\clist_map_inline:nn { fp, int, dim, skip, muskip }
  {
    \cs_generate_variant:cn { #1_set:Nn }  { NV }
    \cs_generate_variant:cn { #1_gset:Nn } { NV }
  }
\ExplSyntaxOff
\documentclass[12pt,hyperref,a4paper,UTF8]{ctexart}
\usepackage{UCASReport}
\setlength{\headheight}{15pt}

%%-------------------------------正文开始---------------------------%%
\begin{document}
\pagenumbering{gobble}

%%-----------------------封面--------------------%%
\cover

%%-----------------------空白页--------------------%%
\newpage
\thispagestyle{empty}
\mbox{}

%%------------------摘要-------------%%
\newpage
\begin{abstract}
片上系统(System-on-Chip, SoC)技术作为现代电子领域的核心,集成了计算、通信、存储等多种功能,为物联网(IoT)、人工智能(AI)和5G通信等领域提供了重要支持。然而,随着SoC设计复杂性和集成度的提升,设计灵活性、验证效率和测试成本等问题日益突出。本文综述了SoC设计与测试领域的最新研究进展,包括Chipyard框架在模块化SoC设计中的应用、电化学阻抗谱(EIS)在电池管理系统实时性能优化中的作用,以及射频SoC测试库设计和IP核验证策略的创新。通过对这些研究的分析,本文总结了现有方法在应对关键问题方面的优势,如设计灵活性提升、验证时间缩短和测试效率优化。综述的内容为SoC技术的未来发展提供了参考方向,也为相关领域的进一步研究奠定了理论基础。


\vspace{1em}

\noindent
\textbf{关键词:} 片上系统,模块化设计,验证方法,电化学阻抗谱,射频测试,IP核集成
\end{abstract}

\thispagestyle{empty} % 首页不显示页码

%%--------------------------目录页------------------------%%
% \newpage
% \tableofcontents

%%------------------------正文页从这里开始-------------------%
\newpage
\pagenumbering{arabic}
\setcounter{page}{1}

%%可选择这里也放一个标题
%\begin{center}
%    \title{ \Huge \bf{{标题}}}
%\end{center}

\section{引言}

片上系统(System-on-Chip, SoC)作为当今半导体技术的核心,广泛应用于无线通信、物联网(IoT)、人工智能(AI)等领域,其设计与验证的复杂性随着集成度的提高而不断增加。在这些技术发展的推动下,研究者们针对SoC设计、测试和验证提出了一系列创新方法与框架,为解决现有问题提供了新思路。例如,\textit{Chipyard: Integrated Design, Simulation, and Implementation Framework for Custom SoCs} 提出了一种模块化生成器框架,显著提高了异构SoC设计的灵活性和验证效率\cite{9099108};而 \textit{EIS-Based SoC Estimation: A Novel Measurement Method for Optimizing Accuracy and Measurement Time} 则通过基于电化学阻抗谱的优化技术,在电池管理系统中实现了实时性能评估的突破\cite{10227269}。

在无线通信领域,\textit{Architecting Millisecond Test Solutions for Wireless Phone RFICs} 通过毫秒级测试方法,显著降低了射频集成电路(RFIC)的生产测试成本,为高效制造提供了新方向\cite{1041873}。与此同时,射频SoC测试的复杂性进一步推动了测试库设计的创新,\textit{Next Generation Test Library for RF SOC on ATE} 引入了现代软件设计模式,为自动化测试提供了灵活而高效的解决方案\cite{9461580}。此外,异构SoC中的IP核验证一直是设计的难点,\textit{Integration and Verification of IP Cores on SoC} 通过通用验证方法学(UVM)框架,在模块化和功能覆盖率上取得了显著进展\cite{9641547}。

本文综述了上述五篇文章的研究内容,分析其研究动机、面临的关键问题及解决方法,并总结了它们在SoC设计与测试领域的创新贡献。这些研究不仅推动了SoC技术的发展,也为未来的设计与验证提出了新的方向。

\section{研究背景与动机}
片上系统(SoC)作为集成电路(IC)领域的重要发展方向,其设计和制造的复杂性伴随着集成度的提高而显著增加\cite{8000621}。在现代电子系统中,SoC已成为计算、通信和存储等功能的核心支撑。随着超大规模集成电路(VLSI)技术的发展,SoC设计的可扩展性和功能多样性为诸多领域提供了新的可能性。然而,SoC的广泛应用不仅依赖于设计本身的创新,也高度依赖于高效的测试和验证方法,以确保其功能的正确性和可靠性。这些技术背景和发展趋势为研究者提出了新的挑战,并成为推动本文所述五篇研究工作的主要动因。

现代SoC设计中面临的一个核心挑战是如何在缩短开发周期的同时保证高质量的设计输出。传统的测试和验证方法难以适应当今复杂系统的需求。例如,在无线射频集成电路(RFIC)的制造中,测试成本和时间已成为主要瓶颈\cite{9904916}。传统自动测试设备(ATE)在处理高频信号测试时,不仅成本昂贵,而且难以实现快速、大批量的制造测试\cite{1041873}。在这一背景下,提出低成本、毫秒级测试解决方案的需求尤为迫切。

此外,随着物联网(IoT)、人工智能(AI)和5G通信等技术的崛起,SoC设计中异构架构的需求日益增加。这种需求不仅体现在硬件模块的多样化,还体现在SoC设计工具链的复杂性。传统工具和验证方法在处理异构系统的动态配置和模块化集成时,显得力不从心。为应对这些挑战,Chipyard框架应运而生,通过模块生成器和自动化仿真工具,支持SoC设计的快速迭代\cite{9099108}。这一框架特别强调通过开源生态系统和FPGA加速仿真降低开发门槛,为学术界和工业界提供了新方法。

除了硬件层面的设计复杂性,许多新兴应用(如电池管理系统)还提出了实时性能评估的新需求。在这些应用中,状态监控的精确性和响应速度至关重要。例如,电池的荷电状态(SoC)直接影响能源管理系统的安全性和效率。然而,以库仑积分法和开路电压法为代表的现有测量方法存在精度不足或响应迟缓的问题。为此,研究者提出基于电化学阻抗谱(EIS)的新方法,通过特征选择和机器学习技术显著缩短测量时间,同时提高精度\cite{10227269}。

在SoC集成和验证中,如何高效地管理和验证大量知识产权(IP)核也是一个持久的难题。随着异构系统设计的普及,IP核之间的通信需求不断增长,接口标准的多样性进一步加剧了集成难度。在此背景下,基于通用验证方法学的验证策略提供了新的解决思路。借助模块化测试组件的复用,这种方法显著提高了验证效率和覆盖率\cite{9641547}。

与此同时,射频SoC测试的特殊性为测试库设计提出了新要求。在射频SoC中,由于涉及多种无线标准和频率范围,测试方法的复杂性和硬件资源消耗均远高于传统数字芯片。新一代测试库通过引入现代软件设计模式(如策略模式和模板方法),不仅简化了测试流程,还提高了测试的自动化程度和执行效率。这种方法尤其适合快速迭代的射频应用场景\cite{9461580}。

以上文章的研究动机均源于SoC设计与测试领域的实际需求,包括测试成本、验证效率、设计复杂性和实时性能评估的权衡。这些研究工作在不同层次和领域上提出了新的解决方案,为SoC设计和测试的未来发展提供了新的思路和方法。

\section{关键问题}

片上系统(SoC)设计、测试和验证领域的核心问题在于如何应对复杂性和高效性之间的矛盾。随着集成度的提高和异构架构的普及,传统方法在性能和成本上的限制变得更加明显。这些问题贯穿于SoC的设计、验证和应用的各个环节,不同场景中具有各自的独特挑战。

\subsection{异构SoC设计的集成复杂性}
Chipyard框架的研究揭示了当前异构SoC设计中的主要挑战。异构SoC集成了多种IP模块,包括处理器、存储器、通信接口和加速器单元。由于模块功能和架构的多样性,其集成复杂性显著增加。传统的硬件设计方法在面对这些需求时显得力不从心,尤其是如何在功能多样化和验证效率之间找到平衡成为关键问题。

Chipyard框架通过模块化设计生成器实现了异构模块的快速集成,但其研究也表明,随着模块数量和复杂度的增加,参数化设计和验证的扩展性成为新的瓶颈\cite{9099108}。例如,对于多核架构或高度并行的SoC设计,传统的仿真和验证手段难以充分满足实时性需求。尽管FPGA加速仿真在一定程度上缓解了验证时间的问题,但云端仿真与实际硬件之间仍存在较大差距,这对高精度设计提出了更高要求。

\subsection{电池管理系统中的实时性能评估难题}
在电池管理系统(BMS)中,荷电状态(SoC)估算的实时性与精度需求一直是关键问题。传统方法,如库伦积分法和开路电压法,无法在动态环境中同时满足两者的要求。这些方法受限于测量精度和响应速度,其应用范围被局限于低频、静态场景,而动态、高速变化的实际应用中难以发挥有效作用\cite{10227269}。

Bourelly等的研究通过基于电化学阻抗谱(EIS)的方法显著提升了估算性能。然而,其研究中也指出,频率选择和特征优化对模型性能至关重要,而高效实现这一过程的技术仍然不足。例如,EIS中低频信号的测量时间通常较长,这使得实时性的提升需要以牺牲部分特征为代价。如何在频率分布和特征选择上达到最优平衡,以及如何适配更广泛的电池类型,是这一领域继续探索的难题\cite{10227269}。

\subsection{射频SoC测试的复杂性}
射频SoC测试的关键问题在于如何在支持多无线标准和多频段的情况下,保持测试的高效性和可靠性。Ping Wang等的研究揭示了传统射频测试方法的局限性:单一任务的测试流程耗时长,工程师需要大量时间熟悉复杂的测试库结构,而每次新技术的引入都会显著增加测试成本\cite{9461580}。

尽管新一代测试库通过模块化和多线程设计在一定程度上解决了这些问题,但研究也表明,射频SoC的快速更新换代对测试库的扩展能力提出了更高要求。例如,对于5G、毫米波雷达和IoT设备等新兴应用场景,如何在测试效率和测试覆盖率之间找到平衡仍然是一个未解难题。特别是在大规模生产中,调试和引入新功能测试的时间成本始终是优化的重点。

\subsection{IP核验证的接口问题与覆盖率难题}
异构SoC中的IP核验证问题集中在接口的正确性和覆盖率的提升上。Naik等指出,随着SoC架构的复杂性增加,接口不匹配和信号冲突成为验证中的主要障碍\cite{9641547}。传统验证方法不仅耗时长,而且覆盖率较低,这对SoC的功能完整性提出了挑战。

UVM框架通过模块化的验证结构在一定程度上缓解了这些问题,但仍面临以下关键难点:一是复杂通信协议的验证需要更多的资源支持,二是多模块间交互的验证需求快速增长,导致验证任务呈指数级增加。此外,如何提高验证组件的复用性,减少开发人员的重复工作量,也是当前方法需要进一步优化的方向\cite{9641547}。



\section{技术方案综述}

片上系统(SoC)的设计与验证技术逐步演进,研究者提出了多种创新解决方案以应对异构架构、验证效率和实时性能需求的挑战。以下介绍几个具有代表性的解决方法:Chipyard框架侧重于SoC的设计灵活性和验证效率;基于电化学阻抗谱(EIS)的优化方法解决了电池管理系统(BMS)中的实时性能问题;此外,针对射频SoC的测试复杂性和IP核验证的技术难点,也提出了卓有成效的方案。

\subsection{敏捷设计与验证框架——Chipyard}
Chipyard框架由Amid等提出,是一种以模块化设计和自动化验证工具为核心的敏捷开发与验证平台\cite{9099108}。该框架以开源为基础,结合了硬件生成器和多层次仿真技术,特别适用于异构SoC的快速开发与验证。它的独特之处在于通过参数化设计支持多种IP模块的动态配置,使SoC的设计与验证得以高度集成。

Chipyard的设计生成器基于Chisel语言,通过参数化描述实现灵活的模块化设计。相比传统的硬件描述语言,该方法更加易于扩展和定制。在验证层面,该框架提供了云端FPGA加速仿真工具,可以大幅降低验证的时间成本。此外,Chipyard通过中间表示语言FIRRTL支持从前端设计到后端物理实现的全流程优化,确保了高效的硬件资源利用率。实验表明,该框架不仅适用于学术研究中的多核处理器设计,还能满足工业界对高复杂度SoC的验证需求\cite{9099108}。

\subsection{基于EIS的实时性能优化方法}
在BMS中,荷电状态的实时评估是关键问题之一。Bourelly等通过研究电化学阻抗谱的频率响应特性,提出了一种全新的优化测量方法\cite{10227269}。传统的库伦积分法和开路电压法难以满足动态场景下对精度和实时性的双重需求,而EIS方法通过特征选择和机器学习技术为这一领域提供了突破。

具体而言,EIS方法首先通过频率响应采集电池的阻抗数据。这些数据通常包含大量冗余信息,为此,研究者引入特征选择技术以确定最重要的频率特征。机器学习模型(如支持向量机)被用于分类荷电状态,同时优化分类器的性能以确保高效性和鲁棒性。实验结果显示,特征选择技术使测量时间从4分钟缩短至1.5秒,同时分类精度从0.87提升至0.98。该方法的另一亮点在于通过排除误导特征显著提升了模型的泛化能力,为实时性能监测提供了一种有效的解决方案。

这一优化过程结合了实验数据驱动的分析和算法优化,不仅适用于BMS领域,还具有很强的通用性,可以推广到需要实时评估的其他技术场景。通过降低计算复杂度和测量时间,EIS方法在平衡性能与效率方面展现了独特优势。

\subsection{射频SoC测试库设计}
射频SoC测试的复杂性主要体现在多频段和多无线标准的支持上,同时测试时间长、工程师上手难度大等问题也限制了测试效率。为应对这些挑战,Ping Wang等提出了新一代测试库设计\cite{9461580}。该测试库通过引入现代软件设计模式实现了测试流程的高度模块化和自动化,同时支持多线程优化和灵活扩展。

测试库的核心结构基于策略模式和模板方法,这使得用户可以通过参数化配置快速适配不同射频测试需求。在硬件配置方面,测试库提供了统一的接口以支持寄存器设置、数据采集和信号调制等功能。尤其在5G等新兴技术中,该测试库通过并行处理多项测试任务,显著提升了执行效率。此外,测试库的模块化设计降低了新工程师的学习成本,新用户可以在短时间内完成复杂测试程序的开发。

实验结果表明,新一代测试库的效率较传统方法提升了约30\%,同时支持快速适应未来技术的测试标准扩展。这种灵活且高效的设计为射频SoC的自动化测试设定了新的标准。

\subsection{基于UVM的IP核验证技术}
IP核的集成验证是异构SoC设计中的核心环节。Naik等通过研究IP核通信的复杂性,采用了通用验证方法学(UVM)框架\cite{9641547}。与传统验证方法相比,UVM通过标准化测试组件和分层验证流程,显著提升了验证的效率和覆盖率。

UVM框架的独特之处在于其模块化测试架构和覆盖率优化机制。研究中提出的验证流程包括构建模块级测试组件和顶层功能验证,通过IP-XACT标准确保接口连接的正确性。此外,UVM框架支持自动化测试生成和错误检测,使得验证时间大幅缩短,验证覆盖率达到88.35\%。这不仅减少了设计人员在验证中的重复工作量,还显著提升了SoC系统的可靠性。

结合IP核通信的复杂性,UVM验证框架提供了可靠的解决方案,特别是在多模块集成的SoC设计中展现了强大的适应性和扩展性。

\section{总结}

片上系统(SoC)的广泛应用推动了其设计、验证与测试技术的快速演进。然而,面对高集成度和复杂性的现代SoC,传统方法在设计灵活性、验证效率和测试成本等方面的局限性日益显现。本文综述了SoC设计与测试领域的最新进展,围绕模块化设计框架、实时性能优化和自动化测试等关键技术,分析了现有方法的优势与适用性,并提出了未来发展方向。

从整体来看,模块化设计框架的引入极大提升了异构SoC设计的灵活性。通过参数化生成器和多层次仿真工具,研究者不仅显著缩短了开发周期,还为复杂系统的动态配置和快速迭代提供了强有力的支持。尤其是在SoC多功能集成需求日益增长的背景下,模块化设计的优势进一步凸显,为开发高效且可扩展的系统奠定了基础。

实时性能优化方法在动态环境下展现了独特的价值。通过将实验优化与数据驱动分析相结合,研究者成功解决了实时监测中的效率与精度平衡难题。这一技术突破不仅优化了特定领域(如电池管理系统)的性能评估流程,也为其他场景的实时性需求提供了参考和借鉴。

自动化测试和验证工具的发展则在应对测试复杂性和验证效率方面表现出色。模块化测试库设计和多线程技术的结合,不仅显著提高了测试效率,还降低了新技术引入的成本。这种设计方法的灵活性为测试工具的未来扩展性提供了保障。同时,自动化验证框架通过标准化的验证流程和覆盖率优化机制,为复杂系统的功能完整性和可靠性提供了全面保障。

综上所述,现代SoC技术的发展得益于模块化设计、实时优化与自动化验证的紧密结合。这些研究工作不仅解决了SoC技术面临的关键问题,还展示了高效、灵活和可扩展的技术路线。

%%----------- 参考文献 -------------------%%

\reference


\end{document}